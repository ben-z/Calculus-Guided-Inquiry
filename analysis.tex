%----------------------------------------------------------------------------------------
%	ANALYSIS
%----------------------------------------------------------------------------------------

\section*{Analysis of Data}

\subsubsection*{Theoretical}
\label{subs:Theoretical}

To explain the behavior of the ball's acceleration relative to $\theta$, the theoretical data should first be considered. Since gravity, $g$, is approximately $9.8\;m/s^2$ on the surface of the earth, the composite acceleration at any angle relative to horizontal can be found using the following equation:

\begin{equation}
  a = 9.8\sin{\theta}
\end{equation}

\begin{table}[H]
  \centering
  \csvloop{
    file={assets/Data/Theoretical-Angle_Time_Acceleration_Horizontal.csv},
    no head,              % no special treatment of first line
    column count=10,       % since no first line is given, tell about column count
    before reading=\begin{tabular}{|l|c|c|c|c|c|c|c|c|c|}\hline,
    command=\csvlinetotablerow,
    late after line=\\\hline,
    % late after first line=\\\hline,
    % late after last line=\\\hline,
    after reading=\end{tabular}
  }
  \caption{Theoretical Data}
  \label{table:theoretical_data_table}
\end{table}

% Side-by-side graphs
%
\begin{figure}[H]
  \centering
  \begin{minipage}[b]{0.45\linewidth}
    \centering
    \begin{tikzpicture}
      \begin{axis}[
        xlabel={$\theta$ (\textdegree)},
        ylabel={Acceleration ($m/s^2$)},
        xmin = 0, ymin=0,
        scale only axis,
        width=0.8\textwidth,
        y label style={at={(axis description cs:0.08,.5)}},
        ]
        \addplot [black,mark = *] table [header=true, x=Angle, y=Acceleration, col sep=comma] {assets/Data/Theoretical-All_Vertical.csv};
      \end{axis}
    \end{tikzpicture}
    \caption{Acceleration vs $\theta$}
    \label{fig:theoretical_data_theta_plot}
  \end{minipage}
  \quad
  \begin{minipage}[b]{0.45\linewidth}
    \centering
    \begin{tikzpicture}
      \begin{axis}[
        xlabel={$\sin{\theta}$},
        ylabel={Acceleration ($m/s^2$)},
        xmin = 0, ymin=0,
        scale only axis,
        width=0.8\textwidth,
        y label style={at={(axis description cs:0.08,.5)}},
        ]
        \addplot [black,mark = *] table [header=true, x=SinAngle, y=Acceleration, col sep=comma] {assets/Data/Theoretical-All_Vertical.csv};
      \end{axis}
    \end{tikzpicture}
    \caption{Acceleration vs $\sin{\theta}$}
    \label{fig:theoretical_data_sintheta_plot}
  \end{minipage}
\end{figure}
%
% End Side-by-side graphs

When the acceleration vs $\theta$ graph is plotted (Figure \ref{fig:theoretical_data_theta_plot}), the resulting curve looks like a portion of a sinusoidal function. A relationship between acceleration and $\sin{\theta}$ can then be established (Figure \ref{fig:theoretical_data_sintheta_plot}):

\begin{equation}
  a \propto \sin{\theta}
\end{equation}
\\
% Begin Experimental ---------------------
\subsubsection*{Experimental}
\label{subs:Experimental}

With the above relationship, a discussion about the experimental data can now take place. Given Equation:

\begin{equation}
  d = v_0t + \frac{1}{2}at^2
\end{equation}

Above equation can be re-ordered to find the composite acceleration as a function of time ($v_0t$ is 0 because the object starts at rest):

\begin{equation}
  a(t) = \frac{2d}{t^2}
\end{equation}

\begin{table}[H]
  \centering
  \csvloop{
    file={assets/Data/Experimental-Angle_Time_Acceleration_Horizontal.csv},
    no head,              % no special treatment of first line
    column count=10,       % since no first line is given, tell about column count
    before reading=\begin{tabular}{|l|c|c|c|c|c|c|c|c|c|}\hline,
    command=\csvlinetotablerow,
    late after line=\\\hline,
    % late after first line=\\\hline,
    % late after last line=\\\hline,
    after reading=\end{tabular}
  }
  \caption{Experimental data with calculated acceleration}
  \label{table:experimental_data_table_with_acceleration}
\end{table}

% Side-by-side graphs
%
\begin{figure}[H]
  \centering
  \begin{minipage}[b]{0.45\linewidth}
    \centering
    \begin{tikzpicture}
      \begin{axis}[
        xlabel={$\theta$ (\textdegree)},
        ylabel={Acceleration ($m/s^2$)},
        xmin = 0, ymin=0,
        scale only axis,
        width=0.8\textwidth,
        y label style={at={(axis description cs:0.08,.5)}},
        ]
        \addplot [black,mark = *] table [header=true, x=Angle, y=Acceleration, col sep=comma] {assets/Data/Experimental-All_Vertical.csv};
      \end{axis}
    \end{tikzpicture}
    \caption{Acceleration vs $\theta$}
    \label{fig:experimental_data_theta_plot}
  \end{minipage}
  \quad
  \begin{minipage}[b]{0.45\linewidth}
    \centering
    \begin{tikzpicture}
      \begin{axis}[
        xlabel={$\sin{\theta}$},
        ylabel={Acceleration ($m/s^2$)},
        xmin = 0, ymin=0,
        scale only axis,
        width=0.8\textwidth,
        y label style={at={(axis description cs:0.08,.5)}},
        ]
        \addplot [only marks, mark = *] table [header=true, x=SinAngle, y=Acceleration, col sep=comma] {assets/Data/Experimental-All_Vertical.csv};
        \addplot [black] table[
          header=true,
          col sep=comma,
          x=SinAngle,
          y={create col/linear regression={y=Acceleration}},
        ]{assets/Data/Experimental-All_Vertical.csv};
      \end{axis}
    \end{tikzpicture}
    \caption{Acceleration vs $\sin{\theta}$}
    \label{fig:experimental_data_sintheta_plot}
  \end{minipage}
\end{figure}
%
% End Side-by-side graphs

This relationship can still faintly be noticed:

\begin{equation}
  a \propto \sin{\theta}
\end{equation}

For a complete equation, a constant $k$ is needed:

\begin{align}
  a & = k\sin{\theta}\\
  k & = \frac{a}{\sin{\theta}}
\end{align}

A table of values of $k$:

\begin{table}[H]
  \centering
  \csvloop{
    file={assets/Data/Experimental-k_Horizontal.csv},
    no head,              % no special treatment of first line
    column count=10,       % since no first line is given, tell about column count
    before reading=\begin{tabular}{|l|c|c|c|c|c|c|c|c|c|}\hline,
    command=\csvlinetotablerow,
    late after line=\\\hline,
    % late after first line=\\\hline,
    % late after last line=\\\hline,
    after reading=\end{tabular}
  }
  \caption{Experimental Constant}
  \label{table:experimental_k_table}
\end{table}

Friction prevents the number to stay consistent. An average of these constants will be taken to represent the trend.

\begin{equation}
  \frac{6.51 + 7.22 + 7.72 + 7.84 + 10.44 + 12.49 + 12.66 + 12.69 + 13.85}{9} = 10.16
\end{equation}

Therefore, the experimental value of $k$ is $10.16$:

\begin{equation}
  a = 10.16\sin{\theta}
\end{equation}

\subparagraph{Instantaneous Velocity and Position}
\label{subp:Instantaneous Velocity and Position}
At any angle, $a = 10.16\sin{\theta}$ will be a constant. This allows the addition of a new variable---time, $t$.\\

Velocity is the anti-derivative of acceleration, therefore:

\begin{equation}
  v = 10.16\sin{\theta}t
\end{equation}

Position is the anti-derivative of velocity:

\begin{equation}
  s = 5.08\sin{\theta}t^2
\end{equation}

% \subparagraph{A Note on components}
% \label{subp:A Note on components}
% Throughout the experiment, only one force is being applied on the ball---its weight.
%
%
%
% With the given equation, $d = v_{0}t + 2at^2$, the acceleration of the ball
% at each angle can be found as the output of a function of time:
%
% \begin{center}
%   \[
%     a(t) = \fraction{2(d-v_0t)}{t^2}
%   \]
% \end{center}
