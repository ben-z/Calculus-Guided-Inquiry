%%%%%%%%%%%%%%%%%%%%%%%%%%%%%%%%%%%%%%%%%
% University/School Laboratory Report
% LaTeX Template
% Version 3.1 (25/3/14)
%
% This template has been downloaded from:
% http://www.LaTeXTemplates.com
%
% Original author:
% Linux and Unix Users Group at Virginia Tech Wiki
% (https://vtluug.org/wiki/Example_LaTeX_chem_lab_report)
%
% License:
% CC BY-NC-SA 3.0 (http://creativecommons.org/licenses/by-nc-sa/3.0/)
%
%%%%%%%%%%%%%%%%%%%%%%%%%%%%%%%%%%%%%%%%%

%----------------------------------------------------------------------------------------
%	PACKAGES AND DOCUMENT CONFIGURATIONS
%----------------------------------------------------------------------------------------

\documentclass{article}

\usepackage[letterpaper, top=1.3in]{geometry}
\usepackage[version=3]{mhchem} % Package for chemical equation typesetting
\usepackage{siunitx} % Provides the \SI{}{} and \si{} command for typesetting SI units
\usepackage{graphicx} % Required for the inclusion of images
\usepackage{amsmath} % Required for some math elements
\usepackage{enumitem} % Provides custom enumerate numbering
\usepackage{textcomp} % Degree sign
\usepackage{csvsimple} % Tables
\usepackage{pgfplotstable} % Generates table from .csv
\usepackage{pgfplots} % Plot graphs from .csv
\usepackage{float}

\setlength\parindent{0pt} % Removes all indentation from paragraphs

\renewcommand{\labelenumi}{\alph{enumi}.} % Make numbering in the enumerate environment by letter rather than number (e.g. section 6)

%\usepackage{times} % Uncomment to use the Times New Roman font

% Setup siunitx:
\sisetup{
  round-mode          = places, % Rounds numbers
  round-precision     = 2, % to 2 places
}

%----------------------------------------------------------------------------------------
%	DOCUMENT INFORMATION
%----------------------------------------------------------------------------------------

\title{\textbf{Calculus Guided Inquiry} \\ Relationship Between a Function, First Derivative and Second Derivative} % Title

\author{Angus \textsc{Lin}, Alexander \textsc{Froats}, Ben \textsc{Zhang}} % Author name

\date{\today} % Date for the report

\begin{document}

\maketitle % Insert the title, author and date

%----------------------------------------------------------------------------------------
%	Begin Document
%----------------------------------------------------------------------------------------

%----------------------------------------------------------------------------------------
%	OBJECTIVE
%----------------------------------------------------------------------------------------

\section*{Objective}

\begin{enumerate}[label={\arabic*.}]
	% \item{Investigate acceleration, velocity and position in both the horizontal and vertical directions.}
	\item{Experimentally determine an empirical relationship for acceleration, velocity and position as a function of the incline angle for a ball rolling down a ramp.}
\end{enumerate}

% To determine the atomic weight of magnesium via its reaction with oxygen and to study the stoichiometry of the reaction (as defined in \ref{definitions}):
%
% \begin{center}\ce{2 Mg + O2 -> 2 MgO}\end{center}
%
% % If you have more than one objective, uncomment the below:
% \begin{description}
% \item[First Objective] \hfill \\
% Objective 1 text
% \item[Second Objective] \hfill \\
% Objective 2 text
% \end{description}
%
% \subsection{Definitions}
% \label{definitions}
% \begin{description}
% \item[Stoichiometry]
% The relationship between the relative quantities of substances taking part in a reaction or forming a compound, typically a ratio of whole integers.
% \item[Atomic mass]
% The mass of an atom of a chemical element expressed in atomic mass units. It is approximately equivalent to the number of protons and neutrons in the atom (the mass number) or to the average number allowing for the relative abundances of different isotopes.
% \end{description}

%----------------------------------------------------------------------------------------
%	MATERIALS
%----------------------------------------------------------------------------------------

\section*{Materials}

\begin{enumerate}[label={\arabic*.}]
  \item{Ball}
  \item{Ramp}
  \item{Meter stick}
  \item{Protractor}
  \item{Camera}
\end{enumerate}

%----------------------------------------------------------------------------------------
%	PROCEDURE
%----------------------------------------------------------------------------------------

\section*{Procedure}

\begin{enumerate}[label={\arabic*.}]
  \item{Place the ramp at an angle of 10{\textdegree} relative to horizontal
  (using a protractor).}
  \item{Place the meter stick on top of and parallel to the ramp.}
  \item{Set up the camera to face the ramp,
  make sure the entire meter stick can be captured.}
  \item{Place the ball in the slot on the ramp, in alignment with the top
  end of the meter stick.}
  \item{Release the ball, capture the entire footage.}
  \item{Repeat steps 1-5, increase the angle by 10{\textdegree} in every
  capture. Stop after the angle reaches 90{\textdegree}.}
  \item{From the recordings, extract the start time, the time when the ball
  reaches the end of the meter stick, and calculate the difference. Record
  the data with reference to the angle in a table.}
  % \item{With the given formula $d = v_{0}t + 2at^2$, calculate the acceleration
  % for each trial.}
\end{enumerate}

%----------------------------------------------------------------------------------------
%	DATA
%----------------------------------------------------------------------------------------

\section*{Experimental Data}

% \csvautotabular{assets/Data/Experimental-Angle_Time_Horizontal.csv}
\begin{table}[H]
  \centering
  \csvloop{
    file={assets/Data/Experimental-Angle_Time_Horizontal.csv},
    no head,              % no special treatment of first line
    column count=10,       % since no first line is given, tell about column count
    before reading=\begin{tabular}{|l|l|l|l|l|l|l|l|l|l|}\hline,
    command=\csvlinetotablerow,
    late after line=\\,
    late after first line=\\\hline,
    late after last line=\\\hline,
    after reading=\end{tabular}
  }
  \caption{Experimental data}
  \label{fig:experimental_data}
\end{table}

% \begin{tikzpicture}
% \begin{axis}
% \addplot table [x=Angle, y=Acceleration, col sep=comma] {assets/Data/Tables-Vertical.csv};
% \end{axis}
% \end{tikzpicture}

% Begin Table
% \pgfplotstabletypeset[
%   col sep=comma,
%   string type,
%   columns/Angle/.style={column name={$\theta$ (\textdegree)}, column type={|c}},
%   columns/Time/.style={column name={Time ($s$)}, column type={|c}},
%   columns/Acceleration/.style={column name={Acceleration ($m/s^2$)}, column type={|c|}},
%   every head row/.style={before row=\hline,after row=\hline},
%   every last row/.style={after row=\hline},
%   every head row/.append style={
%       typeset cell/.code={
%       \ifnum\pgfplotstablecol=\pgfplotstablecols
%       \pgfkeyssetvalue{/pgfplots/table/@cell content}{\textbf{##1}\\}%
%       \else
%       \pgfkeyssetvalue{/pgfplots/table/@cell content}{\textbf{##1}&}%
%       \fi
%       }
%   },
%   ]{assets/Data/Tables-Vertical.csv}
% End Table

% Side-by-side graphs
%
% \begin{figure}[ht]
%   \centering
%   \begin{minipage}[b]{0.45\linewidth}
%     \centering
%     \begin{tikzpicture}
%       \begin{axis}[
%         xlabel={$\theta$ (\textdegree)},
%         ylabel={Acceleration ($m/s^2$)},
%         scale only axis,
%         width=0.8\textwidth
%         ]
%         \addplot table [header=true, x=Angle, y=Acceleration, col sep=comma] {assets/Data/Tables-Vertical.csv};
%       \end{axis}
%     \end{tikzpicture}
%     \caption{Acceleration vs $\theta$}
%     \label{fig:a_vs_theta}
%   \end{minipage}
%   \quad
%   \begin{minipage}[b]{0.45\linewidth}
%     \centering
%     \begin{tikzpicture}
%       \begin{axis}[
%         xlabel={$sin{\theta}$},
%         ylabel={Acceleration ($m/s^2$)},
%         scale only axis,
%         width=0.8\textwidth
%         ]
%         \addplot table [header=true, x=Angle, y=Acceleration, col sep=comma] {assets/Data/Tables-Vertical.csv};
%       \end{axis}
%     \end{tikzpicture}
%     \caption{Acceleration vs $sin{\theta}$}
%     \label{fig:a_vs_theta}
%   \end{minipage}
% \end{figure}
%
% End Side-by-side graphs

% \begin{tabular}{cccccccccc}
%   $\theta$ ($degrees$) 10 & 20 & 30 & 40 & 50 & 60 & 70 & 80 & 90 \\
%   Time ($s$)
% \end{tabular}
 % Experimental Data
%----------------------------------------------------------------------------------------
%	ANALYSIS
%----------------------------------------------------------------------------------------

\section*{Analysis of Data}

\subsubsection*{Theoretical}
\label{subs:Theoretical}

To explain the behavior of the ball's acceleration relative to $\theta$, the theoretical data should first be considered. Since gravity, $g$, is approximately $9.8\;m/s^2$ on the surface of the earth, the composite acceleration at any angle relative to horizontal can be found using the following equation:

\begin{equation}
  a = 9.8\sin{\theta}
\end{equation}

\begin{table}[H]
  \centering
  \csvloop{
    file={assets/Data/Theoretical-Angle_Time_Acceleration_Horizontal.csv},
    no head,              % no special treatment of first line
    column count=10,       % since no first line is given, tell about column count
    before reading=\begin{tabular}{|l|c|c|c|c|c|c|c|c|c|}\hline,
    command=\csvlinetotablerow,
    late after line=\\\hline,
    % late after first line=\\\hline,
    % late after last line=\\\hline,
    after reading=\end{tabular}
  }
  \caption{Theoretical Data}
  \label{table:theoretical_data_table}
\end{table}

% Side-by-side graphs
%
\begin{figure}[H]
  \centering
  \begin{minipage}[b]{0.45\linewidth}
    \centering
    \begin{tikzpicture}
      \begin{axis}[
        xlabel={$\theta$ (\textdegree)},
        ylabel={Acceleration ($m/s^2$)},
        xmin = 0, ymin=0,
        scale only axis,
        width=0.8\textwidth,
        y label style={at={(axis description cs:0.08,.5)}},
        ]
        \addplot [black,mark = *] table [header=true, x=Angle, y=Acceleration, col sep=comma] {assets/Data/Theoretical-All_Vertical.csv};
      \end{axis}
    \end{tikzpicture}
    \caption{Acceleration vs $\theta$}
    \label{fig:theoretical_data_theta_plot}
  \end{minipage}
  \quad
  \begin{minipage}[b]{0.45\linewidth}
    \centering
    \begin{tikzpicture}
      \begin{axis}[
        xlabel={$\sin{\theta}$},
        ylabel={Acceleration ($m/s^2$)},
        xmin = 0, ymin=0,
        scale only axis,
        width=0.8\textwidth,
        y label style={at={(axis description cs:0.08,.5)}},
        ]
        \addplot [black,mark = *] table [header=true, x=SinAngle, y=Acceleration, col sep=comma] {assets/Data/Theoretical-All_Vertical.csv};
      \end{axis}
    \end{tikzpicture}
    \caption{Acceleration vs $\sin{\theta}$}
    \label{fig:theoretical_data_sintheta_plot}
  \end{minipage}
\end{figure}
%
% End Side-by-side graphs

When the acceleration vs $\theta$ graph is plotted (Figure \ref{fig:theoretical_data_theta_plot}), the resulting curve looks like a portion of a sinusoidal function. A relationship between acceleration and $\sin{\theta}$ can then be established (Figure \ref{fig:theoretical_data_sintheta_plot}):

\begin{equation}
  a \propto \sin{\theta}
\end{equation}
\\
% Begin Experimental ---------------------
\subsubsection*{Experimental}
\label{subs:Experimental}

With the above relationship, a discussion about the experimental data can now take place. Given Equation:

\begin{equation}
  d = v_0t + \frac{1}{2}at^2
\end{equation}

Above equation can be re-ordered to find the composite acceleration as a function of time ($v_0t$ is 0 because the object starts at rest):

\begin{equation}
  a(t) = \frac{2d}{t^2}
\end{equation}

\begin{table}[H]
  \centering
  \csvloop{
    file={assets/Data/Experimental-Angle_Time_Acceleration_Horizontal.csv},
    no head,              % no special treatment of first line
    column count=10,       % since no first line is given, tell about column count
    before reading=\begin{tabular}{|l|c|c|c|c|c|c|c|c|c|}\hline,
    command=\csvlinetotablerow,
    late after line=\\\hline,
    % late after first line=\\\hline,
    % late after last line=\\\hline,
    after reading=\end{tabular}
  }
  \caption{Experimental data with calculated acceleration}
  \label{table:experimental_data_table_with_acceleration}
\end{table}

% Side-by-side graphs
%
\begin{figure}[H]
  \centering
  \begin{minipage}[b]{0.45\linewidth}
    \centering
    \begin{tikzpicture}
      \begin{axis}[
        xlabel={$\theta$ (\textdegree)},
        ylabel={Acceleration ($m/s^2$)},
        xmin = 0, ymin=0,
        scale only axis,
        width=0.8\textwidth,
        y label style={at={(axis description cs:0.08,.5)}},
        ]
        \addplot [black,mark = *] table [header=true, x=Angle, y=Acceleration, col sep=comma] {assets/Data/Experimental-All_Vertical.csv};
      \end{axis}
    \end{tikzpicture}
    \caption{Acceleration vs $\theta$}
    \label{fig:experimental_data_theta_plot}
  \end{minipage}
  \quad
  \begin{minipage}[b]{0.45\linewidth}
    \centering
    \begin{tikzpicture}
      \begin{axis}[
        xlabel={$\sin{\theta}$},
        ylabel={Acceleration ($m/s^2$)},
        xmin = 0, ymin=0,
        scale only axis,
        width=0.8\textwidth,
        y label style={at={(axis description cs:0.08,.5)}},
        ]
        \addplot [only marks, mark = *] table [header=true, x=SinAngle, y=Acceleration, col sep=comma] {assets/Data/Experimental-All_Vertical.csv};
        \addplot [black] table[
          header=true,
          col sep=comma,
          x=SinAngle,
          y={create col/linear regression={y=Acceleration}},
        ]{assets/Data/Experimental-All_Vertical.csv};
      \end{axis}
    \end{tikzpicture}
    \caption{Acceleration vs $\sin{\theta}$}
    \label{fig:experimental_data_sintheta_plot}
  \end{minipage}
\end{figure}
%
% End Side-by-side graphs

This relationship can still faintly be noticed:

\begin{equation}
  a \propto \sin{\theta}
\end{equation}

For a complete equation, a constant $k$ is needed:

\begin{align}
  a & = k\sin{\theta}\\
  k & = \frac{a}{\sin{\theta}}
\end{align}

A table of values of $k$:

\begin{table}[H]
  \centering
  \csvloop{
    file={assets/Data/Experimental-k_Horizontal.csv},
    no head,              % no special treatment of first line
    column count=10,       % since no first line is given, tell about column count
    before reading=\begin{tabular}{|l|c|c|c|c|c|c|c|c|c|}\hline,
    command=\csvlinetotablerow,
    late after line=\\\hline,
    % late after first line=\\\hline,
    % late after last line=\\\hline,
    after reading=\end{tabular}
  }
  \caption{Experimental Constant}
  \label{table:experimental_k_table}
\end{table}

Friction prevents the number to stay consistent. An average of these constants will be taken to represent the trend.

\begin{equation}
  \frac{6.51 + 7.22 + 7.72 + 7.84 + 10.44 + 12.49 + 12.66 + 12.69 + 13.85}{9} = 10.16
\end{equation}

Therefore, the experimental value of $k$ is $10.16$:

\begin{equation}
  a = 10.16\sin{\theta}
\end{equation}

\subparagraph{Instantaneous Velocity and Position}
\label{subp:Instantaneous Velocity and Position}
At any angle, $a = 10.16\sin{\theta}$ will be a constant. This allows the addition of a new variable---time, $t$.\\

Velocity is the anti-derivative of acceleration, therefore:

\begin{equation}
  v = 10.16\sin{\theta}t
\end{equation}

Position is the anti-derivative of velocity:

\begin{equation}
  s = 5.08\sin{\theta}t^2
\end{equation}

% \subparagraph{A Note on components}
% \label{subp:A Note on components}
% Throughout the experiment, only one force is being applied on the ball---its weight.
%
%
%
% With the given equation, $d = v_{0}t + 2at^2$, the acceleration of the ball
% at each angle can be found as the output of a function of time:
%
% \begin{center}
%   \[
%     a(t) = \fraction{2(d-v_0t)}{t^2}
%   \]
% \end{center}
 % Theoretical Data + Experimental D
%-------------------------------------------------------------------------------
%	CONCLUSION
%-------------------------------------------------------------------------------

\section*{Conclusion}

The experimentally determined empirical relationships are as follows:

\begin{itemize}
  \item{
    Acceleration as a function of the incline angle:
    \begin{equation}
      a = 10.16\sin{\theta}
    \end{equation}
  }
  \item{
    Instantaneous Velocity as a function of the incline angle and time:
    \begin{equation}
      v = 10.16\sin{\theta}t
    \end{equation}
  }
  \item{
    Position as a function of the incline angle and time:
    \begin{equation}
      s = 5.08\sin{\theta}t^2
    \end{equation}
  }
\end{itemize}


\end{document}
